\documentclass[letterpaper, 12pt]{article}
\usepackage{../config} 
\date{}

\begin{document}
\begin{titlepage}
\centering
{\bfseries\LARGE Universidad de los Andes \par}
\vspace{1cm}
{\Large Faculty of Engineering \\ Undergraduate Program in Systems Engineering \\ Graduation Project \\ Period 2024-20\par}
\vspace{2.0cm}
\includegraphics[width=0.15\textwidth]{images/Uniandes.png}\par
\vspace{2.0cm}
{\bfseries \LARGE Project Proposal \\ Accessible Automated Test Generation Using Eye-Tracking \par}
\vspace{2.0cm}
\vfill

\vfill
Author: Juan Diego Yepes-Parra \\
Advisor: Camilo Andrés Escobar-Velásquez
\par
\vfill
{ August 2024 \par}
{ Bogotá D.C. \par}
\end{titlepage}

\tableofcontents
\pagebreak

\section{Abstract}

This project explores the potential of eye-tracking technology as a novel means of interaction with computers, particularly focusing on its application in automated test generation for web applications. By harnessing eye movements to control a computer interface and incorporating natural language processing for interpreting user intentions, the project aims to develop an accessible tool that can be of assistance for individuals with motor disabilities. The ultimate goal is to demonstrate that eye-tracking can be effectively used not only as an assistive technology but also as an innovative approach to software testing.
\newline

\noindent\textbf{Keywords:} Eye-tracking, Automated Test Generation, Assistive Technology, Natural Language Processing, Web Applications, Accessibility, Human-Computer Interaction.



\pagebreak

\section{General Objective}


The primary objective of this project is to leverage eye-tracking technology as a user input method for interacting with computers, thus establishing a new mode of human-computer interaction. By combining eye-tracking for pointer control and natural language processing for specifying commands, the project seeks to create an assistive technology that facilitates computer use for individuals with motor impairments. Additionally, the project aims to explore the potential of this interaction model for automating the generation of test scripts for web applications, thereby broadening the scope of its applicability. 


\section{Justification}

This section will detail the rationale behind selecting eye-tracking technology as the focus of the project, without going into details of data and other details, those will be further discussed in the final project document. 
\newline

\noindent The growing interest in inclusive technology and the need for more accessible computing solutions underscore the importance of this research. The project not only seeks to contribute to the field of assistive technologies, but also to introduce a novel approach to software testing. 
\newline

\noindent By providing a dual-purpose tool, the project aims to address both accessibility challenges and the automation needs of software development.
\newline


\section{Specific Objectives}

The specific objectives of this project are as follows:
\begin{enumerate}

\item To develop an \textbf{eye-tracking} system that can be used as an alternative input method for computer interaction.

\item To integrate \textbf{natural language processing} capabilities for interpreting user commands and executing corresponding actions.

\item To design and implement a framework for \textbf{automated test generation} for web applications, utilizing the eye-tracking system as the primary input method. 

\item To evaluate the effectiveness of the developed system in terms of \textbf{accessibility}, \textbf{usability} and accuracy in generating test scripts.


\end{enumerate}

\section{Methodology}

The methodology section will outline the approach taken to achieve the project's objectives. 
\newline

\noindent The project will begin with a comprehensive literature review on eye-tracking technology and its applications in assistive technologies and software testing. 
\newline

\noindent Following this, the development phase will involve the creation of a prototype that integrates eye-tracking with natural language processing. The system will be tested and refined through iterative cycles, with user feedback guiding improvements. 
\newline


\noindent Finally, the project will include a series of experiments to assess the usability and effectiveness of the system in real-world scenarios.

\section{Activity Schedule}

This section will present a detailed timeline of the project, breaking down the tasks and milestones into manageable phases. 

\begin{figure}[h]
\centering
\begin{ganttchart}[
    x unit=0.7cm,
    y unit title=1cm,          % Keep compact height for title row
    y unit chart=0.8cm,          % Keep compact height for the chart rows
    vgrid,
    hgrid,
    title/.append style={fill=blue!12},
    title label font=\bfseries\color{black},
    milestone/.append style={fill=blue},
    bar/.append style={fill=blue},
    bar height=.6,
    bar label font=\scriptsize\color{black},  % Keep label font small
    milestone label font=\scriptsize\color{black},
    group right shift=1,
    group top shift=1,
    group height=1
]{1}{16} % Weeks from 1 to 16

  \gantttitlelist{1,...,16}{1} \\

  \ganttbar{Related Work \& Motivation}{1}{3} \\
  \ganttbar{Eye-tracking}{4}{6} \\
  \ganttbar{Tests}{7}{7} \\
  \ganttbar{Natural Language Processing (NLP)}{8}{10} \\
  \ganttbar{Tests}{11}{11} \\
  \ganttbar{Eye-tracking \& NLP}{12}{13} \\
  \ganttbar{Usability Test}{14}{15} \\
  \ganttbar{Document}{16}{16} \\

\end{ganttchart}
\caption{Project timeline (in weeks)}
\label{fig:gantt}
\end{figure}


\section{Expected Results}

The expected outcomes of this project include the development of a functional eye-tracking-based system for computer interaction, the successful integration of natural language processing for action handling, and the creation of an automated test generation framework for web applications. Additionally, the project aims to produce a comprehensive evaluation of the system's effectiveness in terms of accessibility, usability, and accuracy. Finally, the expected outcome of this project is a possible research contribution that could be published in academic journals, with the goal of generating impactful citations within the fields of assistive technology, software testing, and human-computer interaction. 



\pagebreak



\end{document}