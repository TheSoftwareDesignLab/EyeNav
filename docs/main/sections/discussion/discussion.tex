\section{Discussion}

Overall, the results indicate success in achieving the research objectives. The integration of the eye tracking system with NLP capabilities proved to be an intuitive and responsive method for computer interaction, as demonstrated by user testing. 

However, in terms of usability, the system still requires adjustments to better accommodate diverse user bases. This includes improving accessibility for individuals who wear glasses and addressing challenges faced by users speaking different languages or accents, as these groups reported higher levels of frustration when completing tasks.

On the other hand, the testing module provides a flexible and accessible solution for conducting usability tests on any webpage. Its ease of use, accuracy generating the test and adaptability make it a valuable tool for assessing and improving web interfaces in a dynamic environment.

Finally, this study highlights emerging challenges in web interaction brought about by advancements in AR and VR technologies. As these technologies become more prevalent, it is increasingly important to understand how eye and voice interactions can be effectively utilized in web environments. This research serves as a foundational approach to exploring the usability of such interactions, resolving for future studies to build upon these findings and further enhance user experience in immersive web contexts.
