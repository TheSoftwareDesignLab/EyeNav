\section{Related Work}

\subsection{Eyetracking in a chrome extension for behavioral analysis}

\cite{art:behavioral-analysis-2024} developed a Chrome Extension that leverages eye tracking combined with session recording (also in a record and replay matter) to be able to use for behavior analysis. Their goal was to create a tool that could make behavioral analysis that combine eye tracking and session recording studies easier.

Their extension showed to be highly usable and successful, nevertheless their eye tracking was not as accurate which impacted the usability overall.

This study is interesting in this context because it handles eye tracking combined with a chrome extension and record and replay style sessions, which is very similar to the solution developed.

\subsection{Eyetracking as an input method}

As \cite{art:avp-2024} point out, the recent years have seen a big increment in the usage of eye tracking technologies in consumer-grade AR and VR headsets, such as the Meta Quest Pro, Pico 4 Pro, and Apple Vision Pro. Some of these headsets, like the latter, completely remove the need for physical peripherals, and are controlled using hand gestures and eye tracking exclusively. 

This is an interesting matter of study, because it is very similar to what this study wants to achieve by using eye tracking and voice controls. And even though it can be further tested, \citep{art:avp-2024} found that accuracy is critical for gaze based interaction, but it is not the only influential factor in user experience. 

There are other important considerations for this novel input method. \citep{art:meta-studies-2023} discovered that "a person's natural tendency is not to look at a target, maintain their gaze on that target, and then select. Rather, in natural gaze explorations, we tend to move our eyes quickly from one point of interest to the next", which, as they also point out, makes it harder on the user to estimate exactly when gaze events are ocurring.

Finally, the web can be navigated using eye tracking technologies with the Apple Vision Pro headset. As % TODO poner aqui la referencia de apple
point out, the web needs to be designed and accomodated for these virtual reality environments, since the interaction method is completely different. This is relevant because they found that elements that are bigger and rounder tend to be easier to reach, unlike smaller clickable elements (such as hyperlinks for instance).


\subsection{The future of the web}

\cite{art:web-evolution-revolution-2024} highlights the journey web applications have had over the years, from their static past to now dynamic more complex present. The study also discusses a forward looking perspective on the web, and how it will evolve in AR/VR. A very important point is presented here and that is the multi-modality that future web applications must support, because "Interfaces that combine touch, voice, text, and gesture are expected to rise" \citep{art:web-evolution-revolution-2024}

% TODO mirar si alguna otra referencia me sirva