\section{Related Work}

\subsection{Eyetracking in a chrome extension for behavioral analysis}

\cite{art:behavioral-analysis-2024} developed a Chrome Extension that leverages eye tracking combined with session recording (also in a record and replay matter) to be able to use for behavior analysis. Their goal was to create a tool that could make behavioral analysis that combine eye tracking and session recording studies easier.

Their extension showed to be highly usable and successful, nevertheless their eye tracking was not as accurate which impacted the usability overall.

This study is interesting in this context because it handles eye tracking combined with a chrome extension and record and replay style sessions, which is very similar to the solution developed.

\subsection{Eyetracking as an Input Method}

Recent advancements in consumer-grade AR and VR headsets, such as the Meta Quest Pro, Pico 4 Pro, and Apple Vision Pro, have significantly increased the use of eye tracking technologies \cite{art:avp-2024}. Notably, some of these devices, like the Apple Vision Pro, eliminate the need for physical peripherals, relying solely on hand gestures and eye tracking for control.

This study is relevant as it aims to explore the use of eye tracking and voice controls, similar to the aforementioned technologies. While accuracy is crucial for gaze-based interactions, \cite{art:avp-2024} emphasize that it is not the only factor influencing user experience.

Additional considerations for this novel input method include the natural tendencies of users. \cite{art:meta-studies-2023} found that users do not typically fixate on a target before selecting it. Instead, they quickly shift their gaze from one point of interest to another, complicating the estimation of gaze events.

Furthermore, navigating the web using eye tracking technologies, such as those in the Apple Vision Pro, requires web design adaptations. \cite{video:apple2024spatialweb} highlight that larger, rounder elements are easier to interact with compared to smaller clickable elements like hyperlinks. This insight underscores the need for web interfaces to accommodate virtual reality environments.


\subsection{The future of the web}

\cite{art:web-evolution-revolution-2024} discusses the evolution of web applications from their static origins to their current dynamic and complex state.

The study also provides a forward-looking perspective on the future of the web, particularly in the context of AR/VR. A key point highlighted is the necessity for future web applications to support multi-modality, as "interfaces that combine touch, voice, text, and gesture are expected to rise" \citep{art:web-evolution-revolution-2024}.

Understanding the future of the web is crucial for this study, as it provides insights into usability and potential challenges arising from new interaction methods. This knowledge is invaluable for developers and companies aiming to cater to users in these evolving environments.
