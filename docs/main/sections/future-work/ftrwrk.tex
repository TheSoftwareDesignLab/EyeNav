\section{Future Work}

Considering the scope and impact of the project, the following enhancements and additions can be made to improve the system's accessibility. Furthermore, this study has paved the way for new research areas where the discovered concepts can be applied.

\begin{itemize}
    \item To improve eye tracking for individuals who wear prescription eyeglasses by using their prescription formula to adjust the system for optimal performance.
    \item To expand the system's language support and implement internationalization (i18n) features.
    \item To extend the system capabilities to other environments, such as mobile platforms.
    \item To include a walkthrough page on how to use the system.
    \item To incorporate individuals with motor disabilities in future usability testing sessions, as the system has shown potential to be a beneficial solution based on the system usability scale scores and interview feedback.
    \item To include visual cues for the eyetracking system, to make the system more intuitive
    \item To leverage record-and-replay test generation as a standalone extension, that can be used with these or other interaction methods.
    \item To explore the use of these technologies for testing virtual environments, such as those enabled by Meta Quest or Apple Vision Pro mixed reality glasses. This could involve evaluating the responsiveness and usability of these systems, ensuring accessibility features can accommodate a diverse range of users. 
\end{itemize}
