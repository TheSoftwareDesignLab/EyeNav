\documentclass[10pt, conference]{IEEEtran}
% \IEEEoverridecommandlockouts
% The preceding line is only needed to identify funding in the first footnote. If that is unneeded, please comment it out.
\usepackage{cite}
\usepackage{amsmath,amssymb,amsfonts}
\usepackage{algorithmic}
\usepackage{graphicx}
\usepackage{textcomp}
\usepackage{xcolor}
\def\BibTeX{{\rm B\kern-.05em{\sc i\kern-.025em b}\kern-.08em
    T\kern-.1667em\lower.7ex\hbox{E}\kern-.125emX}}

\input{preamble}
\begin{document}

\title{Paper Title*}

\author{\IEEEauthorblockN{1\textsuperscript{st} Juan Diego Yepes-Parra}
\IEEEauthorblockA{\textit{Universidad de los Andes} \\
Bogotá, Colombia \\
j.yepes@uniandes.edu.co}
\and
\IEEEauthorblockN{2\textsuperscript{nd} Camilo Andrés Escobar-Velásquez}
\IEEEauthorblockA{\textit{Universidad de los Andes} \\
Bogotá, Colombia \\
ca.escobar2434@uniandes.edu.co}
}

\maketitle

\begin{abstract}

Alternative methods for interacting with computers have become increasingly popular, allowing the development of more accessible and intuitive systems. From this idea came EyeNav, a novel system that combines eye tracking and natural language processing (NLP) to enhance accessibility and enable automated test generation. The integration of these technologies for intuitive web interaction, enabling pointer control via gaze and natural language processing for interpreting user intentions, also presents a record-and-replay module for generating automated test scripts. Preliminary user evaluations yielded positive results in terms of usability. The ultimate goal is to demonstrate that this tool effectively used not only as a possible assistive technology but also as an innovative approach to software testing.

% TODO add github and video links

\end{abstract}

\begin{IEEEkeywords}
Eye-tracking; Automated Test Generation; Assistive Technology; Natural Language Processing; Web Applications; Accessibility.
\end{IEEEkeywords}

% must include:
% the envisioned users;
% the software engineering challenge the tool addresses;
% the methodology it implies for its users;
% the results of validation studies already conducted (for mature tools) or - the design of planned studies (for early prototypes).

% A demonstration submission must not exceed four pages (including all text, references, and figures);
% Authors are required to submit a screencast of the tool, with the video link attached to the end of the abstract;
% Authors are encouraged to make their code and datasets open source, with the link for the code and datasets attached to the end of the abstract;

% !TEX root = main.tex

\section{Introduction}

% \authorcomment[missing]{Author}{Missing content note example} \\
% \authorcomment[comment]{Author}{Comment note example} \\
% \authorcomment[note]{Author}{Note example} \\
% \authorcomment[author]{Author}{Author comment note example} \\
% \authorcomment[idea]{Author}{Idea comment note example} \\
% \authorcomment[deleteme]{Author}{DeleteMe comment note example} \\







% !TEX root = main.tex

\section{Related Work}

% \subsection{How acronyms works inside paper}

% \ac{APK} \textit{\textbf{First time called}}

% \ac{APK} \textit{\textbf{everytime ac is called after first one}}

% \acf{APK} \textit{\textbf{Full Acronyms}}

% \acfi{APK} \textit{\textbf{Full Acronym Italic}}

% \acfp{APK} \textit{\textbf{Full Plural Acronym}}

% \acp{APK} \textit{\textbf{Plural Acronym}}

% \acl{APK} \textit{\textbf{Only long name}}

% \aclp{APK} \textit{\textbf{Long name - plural}}

% \textit{\textbf{And more to explore ...}}
% \\
% \\
% \\
Citation \cite{linares15msr}
\input{sections/content}
\input{sections/conclusion}



\section*{Acknowledgment}


\balance
\bibliographystyle{IEEEtran}
\bibliography{local,bib/testing,bib/tools}

\end{document}
